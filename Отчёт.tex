\documentclass[a4paper,12pt,titlepage,finall]{article}

\usepackage{amssymb}
\usepackage{csvsimple}
\usepackage{tikz}
\usepackage{pgfplots}
\usepackage{cmap}
\usepackage{color}
\usepackage[utf8x]{inputenc}
\usepackage[english,russian]{babel}
\usepackage[T2A]{fontenc}
\newcommand{\gt}{\textgreater} % знак больше
\newcommand{\lt}{\textless}       % знак меньше]
\usepackage[margin=2cm]{geometry}		 % для настройки размера полей
\usepackage{indentfirst}         % для отступа в первом абзаце секции
\usepackage{amsmath}
\usepackage{fancyvrb}
\usepackage{graphicx}
\usepackage{listings}



\usepackage{listings}

\lstset{
inputencoding=utf8x,
extendedchars=false,
keepspaces = true,
language=C++,
basicstyle=\ttfamily,
keywordstyle=\color[rgb]{0,0,1},
commentstyle=\color[rgb]{0.026,0.112,0.095},
stringstyle=\color[rgb]{0.627,0.126,0.941},
numberstyle=\color[rgb]{0.205, 0.142, 0.73},
morecomment=[l][\color{magenta}]{\#},
frame=shadowbox,
escapechar=`,
numbers=left,
breaklines=true,
basicstyle=\ttfamily,
literate={\ \ }{{\ }}1,
tabsize=2,
basicstyle=\footnotesize,
}


% выбираем размер листа А4, все поля ставим по 3см
\geometry{a4paper,left=10mm,top=10mm,bottom=10mm,right=10mm}

\setcounter{secnumdepth}{0}      % отключаем нумерацию секций


\begin{document}
% Титульный лист
\begin{titlepage}
    \begin{center}
	{\small \sc Московский государственный университет \\имени М.~В.~Ломоносова\\
	Факультет вычислительной математики и кибернетики\\}
	\vfill
	{\Large \sc Компьютерный практикум по учебному курсу ""}\\
	~\\
	{\large \bf <<Введение в численные методы>> \\
	Задание 2}\\
	~\\
	{\small \bf  Числнные методы решения дифференцальных уравнений\\ }
	~\\ 
	{\large \bf  ОТЧЕТ \\ }
	~\\
	{\small \bf  о выполненном задании \\ }
	~\\
	{\small \sc студента 203 учебной группы факультета ВМК МГУ\\}

	{\small \sc Травниковой Арины Сергеевны\\}
	\vfill
    \end{center}

    \begin{center}
	\vfill
	{\small гор. Москва\\2018 г.}
    \end{center}
\end{titlepage}

% Автоматически генерируем оглавление на отдельной странице
\tableofcontents

\newpage

\section{Постановка задачи}

\subsection{Часть 1}
Рассматривается ОДУ первого порядка, разрешённое относительно производной и имеющее вид, c дополнительным начальным условием в точка a:
	\begin{equation*}
	\begin{cases}
	 \frac{dy}{dx} = f(x, y) &  a \leq x \leq b  \\
	 y(a) = y_{0} 
	\end{cases}
	\end{equation*}
Необходимо найти решение данной задачи Коши в предположении, что правая часть уравнения $f = f(x, y)$ таковы, что гарантирует существование и единственность решения задачи Коши.

\subsection{Часть 2}
Рассматривается система линейных ОДУ первого порядка, разрешённых относительно производной, с дополнительными условиями в точке а :
\begin{equation*}
\begin{cases}
\frac{dy_{1}}{dx} = f_{1}(x, y_{1}, y_{2})\\
\frac{dy_{2}}{dx} = f_{2}(x, y_{1}, y_{2})
 &  a \leq x \leq b  \\
y_{1}(a) = y_{1}^{0},  y_{2}(a) = y_{2}^{0}
\end{cases}
\end{equation*}
Необходимо найти решение данной задачи Коши  в предположении, что правые части уравнений таковы, что гарантируют существование и единственность решения задачи Коши для системы.

\subsection{Часть 3}
Рассматривается  краевая задача для дифференциального уравнения второго порядка с дополнительными условиями в граничных точках:
\begin{equation*}
\begin{cases}
y"+p(x)y'+q(x)y = f(x) \\
\sigma_{1}y(a) + \gamma_{1}y'(a) = \delta_{1} &  a \leq x \leq b  \\
\sigma_{2}y(b) + \gamma_{2}y'(b) = \delta_{2}
\end{cases}
\end{equation*}
Необходимо найти решение данной краевой задачи.


\newpage

\section{Цели}


\begin{itemize}

\item Часть 1

Изучить методы Рунге-Кутта второго и четвертого порядка точности, применяемые для численного решения задач Коши для дифференциального уравнения (или системы) первого порядка:

\begin{itemize}
\item Решить заданные задачи Коши для методоми Рунге-Кутта второго и четвертого порядка точности, апроксимировав дифференциальну задачу соответствующей разностной схемой на равномерной сетке; полученное конечно-разностное уравнение просчитать численно
\item Найти численное решение и построить его график
\item Сравнить численное решение с точным на различных тестах, используя wolframalpha.com
\end{itemize}

\item Часть 2

Изучить метод прогонки решения краевой задачи для дифференциального уравнения второго порядка:

\begin{itemize}
\item Решить краевую заданную задачу методом конечных разностей, аппроксимировав ее разностной схеой второго порядка точности на равномерной сетке; полученную систему конечно-разностных уранвений решить методом прогонки
\item Найти численное решение задачи и построить его график
\item  Сравнить численное решение и с точным на различных тестах, используя wolframalpha.com
\end{itemize}

\end{itemize}

\newpage


\section{Описание алгоритмов}

\subsection{Часть 1}
Будем использовать следующие формулы для численного решения задачи Коши, приближающие точное решение с четвёртым порядком точности относительно диаметра разбиения отрезка, на котором решается поставленная задача.\\
Положим:\\
\begin{itemize}
\item $n$ - число точек разбиения отрезка\\
\item $ h=\frac{a - b}{n}$ - диаметр разбиения отрезка\\
\item $ x_{i} = a + h * i, y_{i} = y(x_{i}), 0 \leq i \leq n $ - сетка и сеточная функция\\
\end{itemize}

Метод Рунге-Кутты 2 порядка точности для рекуррентного вычисления сеточной функции примет следующий вид:
~\\

			$y_{i + 1} = y_{i} + \frac{h}{2}(f(x_{i} + f(x_{i} + h, y_{i} + h * f(x_{i}))))\\$




Метод Рунге-Кутты 4 порядка точности для рекуррентного вычисления сеточной функции примет следующий вид:
	\begin{equation*}
		\begin{cases}
			k_{1} = f(x_{i}, y_{i})\\
			k_{2} = f(x_{i} + \frac{h}{2}, y_{i} + \frac{h}{2}k_{1})\\
			k_{3} = f(x_{i} + \frac{h}{2}, y_{i} + \frac{h}{2}k_{2})\\
			k_{4} = f(x_{i} + h, y_{i} + hk_{3})\\
			y_{i + 1} = y_{i} + \frac{h}{6}(k_{1} + 2k_{2} + 2k_{3} + k_{4})
		\end{cases}
\end{equation*}


\newpage

\subsection{Часть 2}

Метод Рунге-Кутты 2 порядка для рекуррентного вычисления сеточной функции примет следующий вид:
\begin{equation*}
\begin{cases}
k_{1} = f(x_{i}, y^{i}_{1}, y^{i}_{2})\\
k_{2} = f(x_{i} + \frac{h}{2}, y^{i}_{1} + \frac{h}{2}k_{1}, y^{i}_{2} + \frac{h}{2}k_{21})\\
k_{21} = f(x_{i}, y^{i}_{1}, y^{i}_{2})\\
k_{22} = f(x_{i} + \frac{h}{2}, y^{i}_{1} + \frac{h}{2}k_{1}, y^{i}_{2} + \frac{h}{2}k_{21})\\

y^{i + 1}_{1} = y^{i}_{1} + hk_{2} \\
y^{i + 1}_{2} = y^{i}_{2} + hk_{22})
\end{cases}
\end{equation*}

Метод Рунге-Кутты 4 порядка для рекуррентного вычисления сеточной функции примет следующий вид:
\begin{equation*}
\begin{cases}
k_{1} = f(x_{i}, y^{i}_{1}, y^{i}_{2})\\
k_{2} = f(x_{i} + \frac{h}{2}, y^{i}_{1} + \frac{h}{2}k_{1}, y^{i}_{2} + \frac{h}{2}k_{21})\\
k_{3} = f(x_{i} + \frac{h}{2}, y^{i}_{1} + \frac{h}{2}k_{2}, y^{i}_{2} + \frac{h}{2}k_{22})\\
k_{4} = f(x_{i} + h, y^{i}_{1} + hk_{3}, y^{i}_{2} + hk_{23})\\
k_{21} = f(x_{i}, y^{i}_{1}, y^{i}_{2})\\
k_{22} = f(x_{i} + \frac{h}{2}, y^{i}_{1} + \frac{h}{2}k_{1}, y^{i}_{2} + \frac{h}{2}k_{21})\\
k_{23} = f(x_{i} + \frac{h}{2}, y^{i}_{1} + \frac{h}{2}k_{2}, y^{i}_{2} + \frac{h}{2}k_{22})\\
k_{24} = f(x_{i} + h, y^{i}_{1} + hk_{3}, y^{i}_{2} + hk_{23})\\
y^{i + 1}_{1} = y^{i}_{1} + \frac{h}{6}(k_{1} + 2k_{2} + 2k_{3} + k_{4})\\
y^{i + 1}_{2} = y^{i}_{2} + \frac{h}{6}(k_{21} + 2k_{22} + 2k_{23} + k_{24})
\end{cases}
\end{equation*}

\subsection{Часть 3}
Для решения данной задачи запишем заданное дифференциальное уравнение в узлах сетки и краевые условия:\\
\begin{equation*}
\begin{cases}
y''_{i}+p_{i}y'_{i}+q_{i}y_{i}=f_{x}, x_{i} = a + i\frac{b-a}{n} & 0 \leq i \leq n\\
\sigma_{1}y_{0} + \gamma_{1}y'_{0} = \delta_{1}\\
\sigma_{2}y_{n} + \gamma_{2}y'_{n} = \delta_{2}
\end{cases}
\end{equation*}

Для $ 1 \leq i \leq n-1 $ существует следующее разностное приближение для первой и второй производной и самой сеточной функции:\\
\begin{equation*}
\begin{cases}
y''_{i}=\frac{y_{i+1}-2y_{i}+y_{i-1}}{h^{2}}\\
y'_{i}=\frac{y_{i+1}-y_{i-1}}{2h}\\
\end{cases}
\end{equation*}

В результате подстановки этих разностных отношений в начальное уравнение в виде сеточной функции получим линейную систему из n+1 уравнений с n+1 неизвестными $y_0, y_1,...,y_n$:\\
\begin{equation*}
\begin{cases}
y''_{i}=\frac{y_{i+1}-2y_{i}+y_{i-1}}{h^{2}} + p_{i}\frac{y_{i+1}-y_{i-1}}{2h} + q_{i}y_{i} = f_{i} & 1 \leq i \leq n-1\\
\sigma_{1}y_{0} + \gamma_{1}\frac{y_{0}  -y_{0}}{h}= \delta_{1}\\
\sigma_{2}y_{1} + \gamma_{2}\frac{y_{n}  -y_{n-1}}{h}= \delta_{2}
\end{cases}
\end{equation*}

\newpage

Явно выписав коэффициенты перед $y_0, y_1,...,y_n$, получим систему с трехдиагональной матрицей:\\


\bigskip
\begin{minipage}{\linewidth}
	\centering
${A} = 
\begin{bmatrix}
	$$\sigma_{1} -  \gamma_{1} / h $$ & $$\gamma_{1} / h$$ & 0 & 0 & ...  &  \delta_{1}\\
	$$1 - h/2 p_{1}$$ & $$q_{1}*h^{2}-2$$ & $$1+h/2*p_{1}$$ & 0 & ... &  f_{1}h^{2}\\
	$$0$$ & $$1 - h/2 p_{2}$$ & $$q_{2}*h^{2}-2$$ & 1+h/2*p_{2} & ... &  f_{2}h^{2}\\
	0 & 0 & 0 & ... & ... &  f_{k}h^{2}\\
	0 & 0 & 0 & ... & ... &  f_{l}h^{2}\\
	0 &... & $$1 - h/2 p_{n-1}$$ & $$q_{n-1}*h^{2}-2$$ & $$1+h/2*p_{n-1}$$ & $$f_{n-1}h^{2}$$\\
	0 &... & $$0$$ & $$-\gamma_{2}/h$$ & $$\sigma_{2} + \gamma_{2}/h$$ & $$\delta_{2}\\
\end{bmatrix}$
\bigskip
\end{minipage}


Для решения полученной системы используется метод прогонки. Этот метод существенно упрощает решение системы с трехдиагональной матрицей и имеет сложность O(n):\\


Переобозначая коэффициенты перед $y$ получим:
\begin{equation*}
\begin{cases}
C_{0}y_{0} + B_{0}y_{1}=F_{0}\\
A_{i}y_{i-1}+C_{i}y_{i} + B_{i}y_{i+1}=F_{i} & 1 \leq i \leq n-1 \\
A_{n}y_{n-1} + C_{n}y_{n}=F_{n}
\end{cases}
\end{equation*}



\bigskip
\begin{minipage}{\linewidth}
\begin{equation*}
\begin{cases}
\alpha_{0}=-\frac{B_{0}}{C_{0}}\\
\beta_{0}=\frac{F_{0}}{C_{0}}\\
\alpha_{i}=-\frac{B_{i}}{C_{i} + A_{i}\alpha_{i-1}}\\
& 1 \leq i \leq n-1\\
\beta_{i}=\frac{F_{i} - A_{i}\beta_{i-1}}{C_{i} + A_{i}\alpha_{i-1}}\\
\\
y_{n}=\frac{F_{n} - A_{n}\beta_{n-1}}{C_{n} + A_{n}\alpha_{n-1}}\\
y_{i}=\beta_{i}+\alpha_{i} *y_{i+1} & 0 \leq i \leq n-1
\end{cases}
\end{equation*}
\end{minipage}


\newpage

\section{Описание программы}
Рассмотрим ключевые функции программы:\\

\begin{itemize}
\item \begin{verbatim}Matrix.hpp/Matrix.cpp\end{verbatim}

Вычисление определителя:
\begin{lstlisting}

int triangle(double **a, int size) //приведение матрицы к верхнетругольной форме
{
    int j = 0;
    int sign = 1;
    for (int i = 0; i < size; i++){
        // Поиск первого ненулевого элемента в i столбце начиная с i+1 столбца
        for (j = i; j < size; j++){
            if (a[j][i]){
                break;
            }
        }
        if (j == size  || fabs(a[j][i]) < EPS){
            fprintf(stderr,"det 0\n");
            exit(1);
        }
        if (i != j ){
            sign *= change_str(a, i, j); // ф-я меняет местами строки и возвращает -1
        }
        for (int k = i + 1; k < size; k++){
            double k1 = a[i][i];
            double k2 = a[k][i];
            if (fabs(k1) < EPS){
                exit(1);
            }
            if (fabs(k2) < EPS){
                continue;
            }
            for (int l = 0; l < size ; l++){
                a[k][l] -= a[i][l] * k2 / k1;
            }
        }
    }
    return sign;
}

double det(double **a1, int size)
{
    double **a = copy_matrix(a1, size);
    int sign = triangle(a, size);
    double ans = 1;
    for (int i = 0; i < size; i++){
        ans *= a[i][i];
    }
    return ans * sign;
}
\end{lstlisting}
\newpage
Вычисление обратной матрицы:
\begin{lstlisting}
double **inverse(double **a1, int size)
{
    double **a = copy_matrix(a1, size);
    double **res = calloc(size, sizeof(*res));
    for (int i = 0; i < size; i++){
        res[i] = calloc(size, sizeof(*res[i]));
    }
    for (int i = 0; i < size; i++){
        res[i][i]  = 1;
    }
    int j = 0;
    int sign = 1;
    for (int i = 0; i < size; i++){
        // Поиск первого ненулевого элемента в i столбце начиная с i+1 столбца
        for (j = i; j < size; j++){
            if (a[j][i]){
                break;
            }
        }
        if (j == size || fabs(a[j][i]) < EPS){
            fprintf(stderr,"det 0\n");
            exit(1);
        }
        if (i != j){
            sign *= change_str(a, i, j);
            change_str(res, i, j);
        }
        for (int k = i + 1; k < size; k++){
            double k1 = a[i][i];
            double k2 = a[k][i];
            if (fabs(k2) < EPS){
                continue;
            }
            for (int l = 0; l < size ; l++){
                a[k][l] -= a[i][l] * k2 / k1;
                res[k][l] -= res[i][l] * k2 / k1;
            }
        }
    }
    for (int i = size - 1; i > 0; i--){
        for (int j = i - 1; j >= 0; j--){
            if (fabs(a[i][i]) < EPS){
                fprintf(stderr,"det 0\n");
                exit(1);
            }
            double k1 = a[j][i] / a[i][i];
            for (int k = size - 1; k >=0; k--){
                a[j][k] -= k1 * a[i][k];
                res[j][k] -= k1 * res[i][k];
            }
        }
    }
    for (int i = 0; i < size; i++){
        for (int j = 0; j < size; j++){
            if (fabs(a[i][i]) < EPS){
                fprintf(stderr,"det 0\n");
                exit(1);
            }
            res[i][j] /= a[i][i];
        }
    }
    return res;
}

\end{lstlisting}

\newpage

Стандартный метод Гаусса
\begin{lstlisting}
double *gauss(double **a1, double *f1, int size)
{
    double **a = copy_matrix(a1, size);
    double *f = calloc(size, sizeof(*f));
    for (int i = 0; i < size; i++){
        f[i] =f1[i];
    }
    //прямой ход
    int j = 0;
    int sign = 1;
    for (int i = 0; i < size; i++){
        // Поиск первого ненулевого элемента в i столбце начиная с i+1 столбца
        for (j = i; j < size; j++){
            if (a[j][i]){
                break;
            }
        }
        if (j == size  || fabs(a[j][i]) < EPS){
            fprintf(stderr,"det 0\n");
            exit(1);
        }
        if (i != j){
            sign *= change_str(a, i, j);
            double tmp = f[i];
            f[i] = f[j];
            f[j] = tmp;
        }
        for (int k = i + 1; k < size; k++){
            double k1 = a[i][i];
            double k2 = a[k][i];
            if (fabs(k2) < EPS){
                continue;
            }
            f[k] -= f[i] * k2 / k1;
            for (int l = 0; l < size ; l++){
                a[k][l] -= a[i][l] * k2 / k1;
            }
        }

    }
    //обратный ход
    for (int i = size - 1; i > 0; i--){
        for (int j = i - 1; j >= 0; j--){
            if (fabs(a[i][i]) < EPS){
                fprintf(stderr,"det 0\n");
                exit(1);
            }
            double k1 = a[j][i] / a[i][i];
            f[j] -= k1 * f[i];
            for (int k = size - 1; k >=0; k--){
                a[j][k] -= k1 * a[i][k];
            }
        }
    }
    for (int j = 0; j < size; j++){
        if (fabs(a[j][j]) < EPS){
            fprintf(stderr,"det 0\n");
            exit(1);
        }
        f[j] /= a[j][j];
    }
    return f;
}
\end{lstlisting}

\newpage

Метод Гаусса с выбором ведущего элемента.
\begin{lstlisting}
double *gauss_main(double **a1, double *f1, int size)
{
    double **a = copy_matrix(a1, size);
    double *f = calloc(size, sizeof(*f));
    //хранит перестановки столбцов
    int *vec = calloc(size, sizeof(*vec));

    for (int i = 0; i < size; i++){
        f[i] =f1[i];
    }
    //прямой ход
    int j = 0;
    for (int i = 0; i < size; i++){
        vec[i] = i;
    }
    for (int i = 0; i < size; i++){
        // Поиск первого ненулевого элемента в i столбце начиная с i+1 столбца
        double max = 0;
        int idx = 0;
        for (j = i; j < size; j++){
            if (fabs(a[i][j]) > max){
                max = fabs(a[i][j]);
                idx = j;
            }
        }
        double tmp = vec[i];
        vec[i] = vec[idx];
        vec[idx] = tmp;
        for (int l = 0; l < size; l++){
            double tmp = a[l][i];
            a[l][i] = a[l][idx];
            a[l][idx] = tmp;
        }

        for (int k = i + 1; k < size; k++){
            double k1 = a[i][i];
            double k2 = a[k][i];
            f[k] -= f[i] * k2 / k1;
            for (int l = i; l < size ; l++){
                a[k][l] -= a[i][l] * k2 / k1;
            }
        }

    }
    //обратный ход
    for (int i = size - 1; i > 0; i--){
        for (int j = i - 1; j >= 0; j--){
            double k1 = a[j][i] / a[i][i];
            f[j] -= k1 * f[i];
            for (int k = size - 1; k >=0; k--){
                a[j][k] -= k1 * a[i][k];
            }
        }
    }
    for (int j = 0; j < size; j++){
        f[j] /= a[j][j];
    }
    double *ans = calloc(size, sizeof(*ans));
    for (int i = 0; i < size; i++){
        ans[vec[i]] = f[i];
    }
    return ans;
}
\end{lstlisting}
Метод верхней релаксации\\
\begin{lstlisting}
double *relax(double **a, double *f, double w, int size, int max_iter, double solution_eps)
{
    //критерий остановки - максимальное числло итераций
    double *x = calloc(size, sizeof(*x));
    double *tmp = calloc(size, sizeof(*tmp));
    for (int k = 0; k < max_iter; k++){
        for (int i = 0; i < size; i++){
            double sub1 = 0, sub2 = 0;
            for (int j = 0; j < i; j++){
                sub1 += a[i][j] * tmp[j];
            }
            for (int j = i; j < size; j++){
                sub2 += a[i][j] * x[j];
            }
            tmp[i] = x[i] + w / a[i][i] * (f[i] - sub1 - sub2);

            //Проверяем 2 критерий остановки алгоритма - расстояние между векторами решениями
		   на 2 последовательных шагах становится меньше заданного значения - solution_eps
        }
        double dist = 0;
        for (int i = 0; i < size; i++){
            dist += (x[i] -tmp[i]) * (x[i] -tmp[i]);
        }
        if (sqrt(dist) < solution_eps){
            return tmp;
        }
        for (int i = 0; i < size; i++){
            x[i] = tmp[i];
        }
    }
    return x;
}
\end{lstlisting}
Число обусловленности\\
\begin{lstlisting}
double cond_num(double **a, int n)
{
    double res1 = 0;
    for (int i = 0; i < n; i++){
        for (int j = 0; j < n; j++){
            if(fabs(a[i][j]) > res1){
                res1 = fabs(a[i][j]);
            }
        }
    }
    double res2 = 0;
    double **b = inverse(a, n);
    for (int i = 0; i < n; i++){
        for (int j = 0; j < n; j++){
            if(fabs(b[i][j]) > res2){
                res2 = fabs(b[i][j]);
            }
        }
    }
    return res1 * res2;
}

\end{lstlisting}

\newpage


В данном разделе содержится реализация дополнительных функций, использующихся в программе. \\
В частности, реализованы классы:
\begin{itemize}
\item Вывод на стандратный поток матрицы
\begin{lstlisting}
void print_matrix(double **a, int size);
\end{lstlisting}

\item Ввод матрицы со стандратного потока 
\begin{lstlisting}
void scan_matrix(double **a, int size);
\end{lstlisting}

\item Вывод на стандратный поток матрицы-столбца
\begin{lstlisting}
void print_vect(double *a, int size);
\end{lstlisting}

\item Ввод матрицы-столбца со стандратного потока 
\begin{lstlisting}
void scan_vect(double *a, int size);
\end{lstlisting}

\end{itemize}

\end{itemize}

\newpage
\section{Тестирование}

\begin{itemize}


\item Часть 1\\

Будем проводить тестирование на задачах Коши из таблицы 1.

\begin{equation*}
 \begin{cases}
   y' = sin(x) - y\\
   x_0 = 0\\
   y_0 = 10\\
 \end{cases}
\end{equation*}
Точное решение: $$y = -0.5 cosx + 0.5 sinx + 10.5 e^{-x} $$\newline

В этом примере хорошая (невидимая на графике) точность достигается уже при 10 шагах сетки:

\includegraphics[width=450pt]{1.png}\newline


\begin{equation*}
 \begin{cases}
   y' = 3-y-x\\
   x_0 = 0\\
   y_0 = 0\\
 \end{cases}
\end{equation*}
Точное решение: $$y =4 -x - 4 e^{-x} $$\newline

\includegraphics[width=450pt]{2.png}


\begin{equation*}
 \begin{cases}
   y' = -x^{2} - y\\
   x_0 = 0\\
   y_0 = 10\\
 \end{cases}
\end{equation*}
Точное решение: $$y =-x^{3} + 2x -2 + 12 e^{-x} $$\newline


\includegraphics[width=450pt]{3.png}\newline


\newpage
\item Часть 2\\

Будем проводить тестирование на задачах Коши из таблицы 2.


\begin{equation*}
 \begin{cases}
   u' = \frac{u - v}{x}\\
   v' = \frac{u + v}{x}\\
   x_0 = 1\\
   u_0 = 1\\
   v_0 = 1\\
 \end{cases}
\end{equation*}
Точное решение: $$u_{precise}(x) = x(cos(ln(x))-sin(ln(x)))\\
	v_{precise}(x) = x(cos(ln(x))+sin(ln(x))) $$ \newline


\includegraphics[width=450pt]{4.png}\newline
\newpage

\begin{equation*}
 \begin{cases}
   u' =-2*x*u^{2}+v^{2}-x-1\\
   v' = \frac{1}{v^{2}}-u-\frac{x}{u}\\
   x_0 = 0\\
   u_0 = 1\\
   v_0 = 1\\
 \end{cases}
\end{equation*}

\includegraphics[width=450pt]{5.png}\newline

\begin{equation*}
 \begin{cases}
   u' = x*u+v\\
   v' = u-v\\
   x_0 = 0\\
   u_0 = 0\\
   v_0 = 1\\
 \end{cases}
\end{equation*}

\includegraphics[width=450pt]{6.png}\newline

\newpage

\item Часть 3\\

\begin{equation*}
 \begin{cases}
  y''+y=4*sin(x)\\
  y_{0}+y'_{0}=2\\
  y_{1}+y'_{1}=0\\
 \end{cases}
\end{equation*}
Точное решение: $$y =sin(x)+(3-2x)cos(x)$$ \newline


\includegraphics[width=450pt]{7.png}\newline
\newpage

\begin{equation*}
 \begin{cases}
  y''+2y'+y=1\\
  y_{0}=1\\
  y_{1}+y'_{1}=0\\
 \end{cases}
\end{equation*}

Точное решение: $$y =1-xexp{1-x}$$ \newline

\includegraphics[width=450pt]{8.png}\newline

\begin{equation*}
 \begin{cases}
  y''+2y'-\frac{y}{x}=3\\
  y_{0.2}=2\\
  0.5y_{0.5}-y'_{0.5}=1\\
 \end{cases}
\end{equation*}

\includegraphics[width=450pt]{9.png}\newline


\end{itemize}

\newpage
\section{Выводы}

\begin{itemize}
\item Часть 1\\

 Из теоретического материала следует, что  метод Гаусса с выбором главного элемента имеет более высокую точность, однако на примерах, рассмотренных в ходе проведения тестирования, оба метода сходятся и значимое различие в точности результатов не обнаруживается. Вместе с этим методы одновременно трубуют невырожденности матрицы А.Таким образом, можем сказать, что для подобных задач (размер матрицы не велик, сами элементы не являются большими в абсолютном значении) значительной разницы в применении методов не будет.

\item Часть 2\\

В случае симметрической положительно определенной матрицы метод верхней релаксации будет сходится очень быстро, особенно при $\omega = 1$ (Метод Зейделя). В этом случае для получения решения точности порядка $10^{-6}$ метод верхней релаксации потребует порядка $10-30$ итераций в зависимости от $\omega$, что показывает превосходство данного метода над методом Гаусса в плане скорости. Однако для быстрой сходимости он накладывает на матрицу и много ограничений: матрица А должна быть симетрической, положительно определённой, с небольшим числом обусловленности (близким к 1).


\end{itemize}

\newpage


\end{document}