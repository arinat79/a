\documentclass[a4paper,12pt,titlepage,finall]{article}

\usepackage{cmap}
\usepackage{color}
\usepackage[utf8x]{inputenc}
\usepackage[english,russian]{babel}
\usepackage[T2A]{fontenc}
\newcommand{\gt}{\textgreater} % знак больше
\newcommand{\lt}{\textless}       % знак меньше]
\usepackage{geometry}		 % для настройки размера полей
\usepackage{indentfirst}         % для отступа в первом абзаце секции
\usepackage{amsmath}
\usepackage{fancyvrb}



\usepackage{listings}

\lstset{
inputencoding=utf8x,
extendedchars=false,
keepspaces = true,
language=C++,
basicstyle=\ttfamily,
keywordstyle=\color[rgb]{0,0,1},
commentstyle=\color[rgb]{0.026,0.112,0.095},
stringstyle=\color[rgb]{0.627,0.126,0.941},
numberstyle=\color[rgb]{0.205, 0.142, 0.73},
morecomment=[l][\color{magenta}]{\#},
frame=shadowbox,
escapechar=`,
numbers=left,
breaklines=true,
basicstyle=\ttfamily,
literate={\ \ }{{\ }}1,
tabsize=2,
basicstyle=\footnotesize,
}


% выбираем размер листа А4, все поля ставим по 3см
\geometry{a4paper,left=10mm,top=10mm,bottom=10mm,right=10mm}

\setcounter{secnumdepth}{0}      % отключаем нумерацию секций


\begin{document}
% Титульный лист
\begin{titlepage}
    \begin{center}
	{\small \sc Московский государственный университет \\имени М.~В.~Ломоносова\\
	Факультет вычислительной математики и кибернетики\\}
	\vfill
	{\Large \sc Компьютерный практикум по учебному курсу ""}\\
	~\\
	{\large \bf <<Введение в численные методы \\
	Задание 1>>}\\
	~\\
	{\large \bf  ОТЧЕТ \\ }
	~\\
	{\small \bf  о выполненном задании \\ }
	~\\
	{\small \sc студента 203 учебной группы факультета ВМК МГУ\\}

	{\small \sc Травниковой Арины Сергеевны\\}
	\vfill
    \end{center}

    \begin{center}
	\vfill
	{\small гор. Москва\\2018 г.}
    \end{center}
\end{titlepage}

% Автоматически генерируем оглавление на отдельной странице
\tableofcontents
\newpage

\section{Цели}


\begin{itemize}

\item Часть 1

Изучить классическую метод Гаусса (а также модифицированный метод Гаусса), применяемый для решения сислетмы линейных алгебраических уравнений:

\begin{itemize}
\item Решить заданную СЛАУ  методом Гаусса и методом Гаусса с выбором главного элемента
\item Вычислить определителю матрицы $det(A)$
\item Вычислить обратную матрицу  $A ^ {-1}$
\item Определить число обусловленности $M_A = ||A|| * ||A ^ {-1}||$
\item Исследовать вопрос вычислительной  устойчивости метода Гаусса
\item Проверить правильность решения СЛАУ на различных тестах, используя wolframalpha.com
\end{itemize}

\item Часть 2

Изучить классические итерационные методы (Зейделя и верхней реалксации), применяемые для решения сислетмы линейных алгебраических уравнений:

\begin{itemize}
\item Решить заданную СЛАУ итерационным методом Зейделя или методом верхней реалксации
\item Разработать критерий остановки итерационного процесса для гарантированного получения приближенного решения исходной СЛАУ с заданной точностью
\item Изучить скорость сходимости итераций к точному решению задачи при различных итреационных параметрах $\omega$
\item Проверить правильность решения СЛАУ на различных тестах, используя wolframalpha.com
\end{itemize}

\end{itemize}
\newpage
-
\section{Постановка задачи}
\begin{itemize}
\item Часть 1

Дана система линейных уравнений $A\overline{x}=\overline{f}$  порядка $n * n$ с невырожденной матрицей $A$. Написать программу, решающую СЛАУ заданного пользователем размера методом Гаусса и методом Гаусса с выбором главного элемента.\\
Предусмотреть возможность задания элементов матрицы системы и ее правой части как на входной файле данных, так и задания специальных формул.
~\\

\item Часть 2

Дана система линейных уравнений $A\overline{x}=\overline{f}$  порядка $n * n$ с невырожденной матрицей $A$. Написать программу численного решения СЛАУ заданного пользователем размера, использующую численный алгоритм итерационного метода Зейделя:
$$(D + A ^ {(-)})(x^{k+1}-x^{k}) + Ax^{k} = f),$$
где $D, A  ^ {(-)} $ - соответственно диагональная и нижняя треугольные матрицы, k - номер текущей итерации;\\
в случае использования итерационного метода верхней релакцсации итерационный процесс имеет вид:
$$ (D + \omega A ^ {(-)}) \frac{x^{k+1}-x^{k}}{\omega} + Ax^{k} = f),$$
где $\omega$ - итерационный параметр (при  $\omega = 1$ метод переходит в метод Зейделя).\\
Предусмотреть возможность задания элементов матрицы системы и ее правой части как на входной файле данных, так и задания специальных формул.

\end{itemize}

\newpage


\section{Описание алгоритмов}

\begin{itemize}
\item Часть 1\\
\begin{enumerate}
\item Стандартный метод Гаусса.\\


$\quad$Первый шаг: \\ 
$\quad$ Прямой ход метода Гаусса состоит из n последовательных итераций цикла на каждой из которых происходит изменений как матрицы-левой части ($A^{i-1}\Rightarrow{}A^{i}; A^{0}=A$), так и вектора-правой части (${f}^{i-1}\Rightarrow{}{f}^{i},{f}^0={f}$). На каждой итерации рассматриваем $i$ строку матрицы $A^{i}$. Найдём первый ненулевой элемент в данной строке и поменяем местами $i$ столбец со столбцом, содержащим первый ненулевой элемент (при этом запомним, данную перестановку столбцов). Теперь обратим в $0$ все элементы полученной матрицы, стоящие в $i$ столбце, начиная с $i+1$ строки и вплоть до $n$. Этого можно добиться путём вычитания i строки, умноженной на $\frac{a_{j,i}^{i}}{a_{i,i}^{i}}$ из всех последующих строк $i \lt j \leq n$. При этом производим такие же преобразования в векторе-столбце ${f}^{i}$.\\

$\quad$Второй шаг: \\ 
$\quad$ Oбратный метод Гаусса также состоит из n итераций цикла. На $i$ итерации вычисляется $x^{n-i}$ – $n-i$ компонента вектора-ответа ${x}$. На $i$ шаге рассмотрим $n-i$ строку. К этому моменту все $x^{j}, i \lt j \leq n$ уже вычислены. Тогда положим: $x^{i}=\frac{{f}^{n}_{i} - \sum\limits_{k=i+1}^{n} a^{n}_{i,k}x^{k}}{a^{n}_{i,i}}.$
$\quad$Таким образом, проведя все n итераций цикла, найдем искомый вектор-решение ${x}$.\\

\item Метод Гаусса с выбором ведущего элемента.\\

$\quad$Данный метод аналогичен стандартному методу Гаусса, за исключением того, что на каждом шаге прямого хода выбирается не первый нулевой элемент, а наибольший по модулю элемент в строке. Также для получения вектора-решения необходимо учесть перестановки столбцов, производимые в прямом методе.
~\\

\item Вычисление определителя матрицы.\\
~\\
Проведём прямой ход метода Гаусса. При этом, заведём переменную \texttt{det}:
\begin{itemize}
\item Начальное значение: $\texttt{det} = 1$
\item 	При перестановке столбцов: $\texttt{det}= - \texttt{det}$
\item 	На $i$ итерации: $\texttt{det}=\texttt{det} * a^{i}_{i,i}$
\end{itemize}
Тогда определитель $A$: $\texttt{det}(A)=\texttt{det}*\prod\limits_{k=1}^{n} a^{n}_{i,i}$

\item	Вычисление обратной матрицы. Метод Гаусса-Жордана\\
~\\
Рассмотрим расширенную матрицу $A|E$, где $E$ – единичная матрица размера $n*n$. Модифицируем прямой и обратные ходы стандартного метода Гаусса.
\begin{itemize}
\item	Все операции над строками происходят одновременно в обеих матрицах.
\item	Вместо выбора первого ненулевого элемента в строке на $i$ шаге будем выбирать первый не нулевой элемент среди $a_{j,i}^{i},i \lt j \leq n$ и переставлять соответствующие строки в расширенной матрице.
\item	В обратном ходе метода Гаусса не будем производить вычисление ${x}$, а будем, подобно прямому ходу метода Гаусса обращать в $0$ все элементы, стоящие над элементом с индексом $i,i$ в том же столбце, а затем, разделив $i$ строку расширенной матрицы на $a_{i,i}^{n+i}$, превратим $a_{i,i}^{n+i}$ в единицу.
\end{itemize}
Проведя данный алгоритм, получим, что расширенная матрица $A|E\Rightarrow{}E|A^{-1}$. Тогда взяв правую часть полученной расширенной матрицы получим искомую, обратную матрицу.

\end{enumerate}

\item Часть 2

\begin{enumerate}

\item Метод верхней релаксации\\
Для написания алгоритма воспользуемся явной формулой для пересчёта вектора приближённого решения $x^{k+1}_{i} = x^{k}_{i} + \frac{w}{a_{i,i}}(f_{i} - \sum\limits_{j=1}^{i-1} a_{i,j}x^{k+1}_{j} - \sum\limits_{j=i}^{n} a_{i,j}x^{k}_{j}) ; 1 \leq i \leq n$. \\
В качестве критерия остановки процесса рассмотрим следующее правило: зададим некое $\varepsilon \gt 0$ - точность решения. Будем на каждом щаге итерации вычислять следующую величину:  $\rho({x}^{k + 1}, {x}^{k}) = \sqrt{\sum\limits_{i=1}^{n} ({x}^{k+1}_{i} - {x}^{k}_{i})^2}$. Будем останавливать алгоритм как только $\rho({x}^{k + 1},{x}^{k}) \lt \varepsilon$. Такой критерий гарантирует, что $|{x}^{k+1}_{i} - {x}^{k}_{i}| \lt \varepsilon$. Т.е. изменение координат вектора-решения стало достаточно мало. Помимо этого критерием останова будет служить достижение максимального числа итераций.
\end{enumerate}
\end{itemize}
\newpage

\section{Описание программы}
Рассмотрим ключевые функции программы:\\

\begin{itemize}
\item \begin{verbatim}Matrix.hpp/Matrix.cpp\end{verbatim}

Вычисление определителя:
\begin{lstlisting}

int triangle(double **a, int size) //приведение матрицы к верхнетругольной форме
{
    int j = 0;
    int sign = 1;
    for (int i = 0; i < size; i++){
        // Поиск первого ненулевого элемента в i столбце начиная с i+1 столбца
        for (j = i; j < size; j++){
            if (a[j][i]){
                break;
            }
        }
        if (j == size  || fabs(a[j][i]) < EPS){
            fprintf(stderr,"det 0\n");
            exit(1);
        }
        if (i != j ){
            sign *= change_str(a, i, j); // ф-я меняет местами строки и возвращает -1
        }
        for (int k = i + 1; k < size; k++){
            double k1 = a[i][i];
            double k2 = a[k][i];
            if (fabs(k1) < EPS){
                exit(1);
            }
            if (fabs(k2) < EPS){
                continue;
            }
            for (int l = 0; l < size ; l++){
                a[k][l] -= a[i][l] * k2 / k1;
            }
        }
    }
    return sign;
}

double det(double **a1, int size)
{
    double **a = copy_matrix(a1, size);
    int sign = triangle(a, size);
    double ans = 1;
    for (int i = 0; i < size; i++){
        ans *= a[i][i];
    }
    return ans * sign;
}
\end{lstlisting}
\newpage
Вычисление обратной матрицы:
\begin{lstlisting}
double **inverse(double **a1, int size)
{
    double **a = copy_matrix(a1, size);
    double **res = calloc(size, sizeof(*res));
    for (int i = 0; i < size; i++){
        res[i] = calloc(size, sizeof(*res[i]));
    }
    for (int i = 0; i < size; i++){
        res[i][i]  = 1;
    }
    int j = 0;
    int sign = 1;
    for (int i = 0; i < size; i++){
        // Поиск первого ненулевого элемента в i столбце начиная с i+1 столбца
        for (j = i; j < size; j++){
            if (a[j][i]){
                break;
            }
        }
        if (j == size || fabs(a[j][i]) < EPS){
            fprintf(stderr,"det 0\n");
            exit(1);
        }
        if (i != j){
            sign *= change_str(a, i, j);
            change_str(res, i, j);
        }
        for (int k = i + 1; k < size; k++){
            double k1 = a[i][i];
            double k2 = a[k][i];
            if (fabs(k2) < EPS){
                continue;
            }
            for (int l = 0; l < size ; l++){
                a[k][l] -= a[i][l] * k2 / k1;
                res[k][l] -= res[i][l] * k2 / k1;
            }
        }
    }
    for (int i = size - 1; i > 0; i--){
        for (int j = i - 1; j >= 0; j--){
            if (fabs(a[i][i]) < EPS){
                fprintf(stderr,"det 0\n");
                exit(1);
            }
            double k1 = a[j][i] / a[i][i];
            for (int k = size - 1; k >=0; k--){
                a[j][k] -= k1 * a[i][k];
                res[j][k] -= k1 * res[i][k];
            }
        }
    }
    for (int i = 0; i < size; i++){
        for (int j = 0; j < size; j++){
            if (fabs(a[i][i]) < EPS){
                fprintf(stderr,"det 0\n");
                exit(1);
            }
            res[i][j] /= a[i][i];
        }
    }
    return res;
}

\end{lstlisting}

\newpage

Стандартный метод Гаусса
\begin{lstlisting}
double *gauss(double **a1, double *f1, int size)
{
    double **a = copy_matrix(a1, size);
    double *f = calloc(size, sizeof(*f));
    for (int i = 0; i < size; i++){
        f[i] =f1[i];
    }
    //прямой ход
    int j = 0;
    int sign = 1;
    for (int i = 0; i < size; i++){
        // Поиск первого ненулевого элемента в i столбце начиная с i+1 столбца
        for (j = i; j < size; j++){
            if (a[j][i]){
                break;
            }
        }
        if (j == size  || fabs(a[j][i]) < EPS){
            fprintf(stderr,"det 0\n");
            exit(1);
        }
        if (i != j){
            sign *= change_str(a, i, j);
            double tmp = f[i];
            f[i] = f[j];
            f[j] = tmp;
        }
        for (int k = i + 1; k < size; k++){
            double k1 = a[i][i];
            double k2 = a[k][i];
            if (fabs(k2) < EPS){
                continue;
            }
            f[k] -= f[i] * k2 / k1;
            for (int l = 0; l < size ; l++){
                a[k][l] -= a[i][l] * k2 / k1;
            }
        }

    }
    //обратный ход
    for (int i = size - 1; i > 0; i--){
        for (int j = i - 1; j >= 0; j--){
            if (fabs(a[i][i]) < EPS){
                fprintf(stderr,"det 0\n");
                exit(1);
            }
            double k1 = a[j][i] / a[i][i];
            f[j] -= k1 * f[i];
            for (int k = size - 1; k >=0; k--){
                a[j][k] -= k1 * a[i][k];
            }
        }
    }
    for (int j = 0; j < size; j++){
        if (fabs(a[j][j]) < EPS){
            fprintf(stderr,"det 0\n");
            exit(1);
        }
        f[j] /= a[j][j];
    }
    return f;
}
\end{lstlisting}

\newpage

Метод Гаусса с выбором ведущего элемента.
\begin{lstlisting}
double *gauss_main(double **a1, double *f1, int size)
{
    double **a = copy_matrix(a1, size);
    double *f = calloc(size, sizeof(*f));
    //хранит перестановки столбцов
    int *vec = calloc(size, sizeof(*vec));

    for (int i = 0; i < size; i++){
        f[i] =f1[i];
    }
    //прямой ход
    int j = 0;
    for (int i = 0; i < size; i++){
        vec[i] = i;
    }
    for (int i = 0; i < size; i++){
        // Поиск первого ненулевого элемента в i столбце начиная с i+1 столбца
        double max = 0;
        int idx = 0;
        for (j = i; j < size; j++){
            if (fabs(a[i][j]) > max){
                max = fabs(a[i][j]);
                idx = j;
            }
        }
        double tmp = vec[i];
        vec[i] = vec[idx];
        vec[idx] = tmp;
        for (int l = 0; l < size; l++){
            double tmp = a[l][i];
            a[l][i] = a[l][idx];
            a[l][idx] = tmp;
        }

        for (int k = i + 1; k < size; k++){
            double k1 = a[i][i];
            double k2 = a[k][i];
            f[k] -= f[i] * k2 / k1;
            for (int l = i; l < size ; l++){
                a[k][l] -= a[i][l] * k2 / k1;
            }
        }

    }
    //обратный ход
    for (int i = size - 1; i > 0; i--){
        for (int j = i - 1; j >= 0; j--){
            double k1 = a[j][i] / a[i][i];
            f[j] -= k1 * f[i];
            for (int k = size - 1; k >=0; k--){
                a[j][k] -= k1 * a[i][k];
            }
        }
    }
    for (int j = 0; j < size; j++){
        f[j] /= a[j][j];
    }
    double *ans = calloc(size, sizeof(*ans));
    for (int i = 0; i < size; i++){
        ans[vec[i]] = f[i];
    }
    return ans;
}
\end{lstlisting}
Метод верхней релаксации\\
\begin{lstlisting}
double *relax(double **a, double *f, double w, int size, int max_iter, double solution_eps)
{
    //критерий остановки - максимальное числло итераций
    double *x = calloc(size, sizeof(*x));
    double *tmp = calloc(size, sizeof(*tmp));
    for (int k = 0; k < max_iter; k++){
        for (int i = 0; i < size; i++){
            double sub1 = 0, sub2 = 0;
            for (int j = 0; j < i; j++){
                sub1 += a[i][j] * tmp[j];
            }
            for (int j = i; j < size; j++){
                sub2 += a[i][j] * x[j];
            }
            tmp[i] = x[i] + w / a[i][i] * (f[i] - sub1 - sub2);

            //Проверяем 2 критерий остановки алгоритма - расстояние между векторами решениями
		   на 2 последовательных шагах становится меньше заданного значения - solution_eps
        }
        double dist = 0;
        for (int i = 0; i < size; i++){
            dist += (x[i] -tmp[i]) * (x[i] -tmp[i]);
        }
        if (sqrt(dist) < solution_eps){
            return tmp;
        }
        for (int i = 0; i < size; i++){
            x[i] = tmp[i];
        }
    }
    return x;
}
\end{lstlisting}
Число обусловленности\\
\begin{lstlisting}
double cond_num(double **a, int n)
{
    double res1 = 0;
    for (int i = 0; i < n; i++){
        for (int j = 0; j < n; j++){
            if(fabs(a[i][j]) > res1){
                res1 = fabs(a[i][j]);
            }
        }
    }
    double res2 = 0;
    double **b = inverse(a, n);
    for (int i = 0; i < n; i++){
        for (int j = 0; j < n; j++){
            if(fabs(b[i][j]) > res2){
                res2 = fabs(b[i][j]);
            }
        }
    }
    return res1 * res2;
}

\end{lstlisting}

\newpage


В данном разделе содержится реализация дополнительных функций, использующихся в программе. \\
В частности, реализованы классы:
\begin{itemize}
\item Вывод на стандратный поток матрицы
\begin{lstlisting}
void print_matrix(double **a, int size);
\end{lstlisting}

\item Ввод матрицы со стандратного потока 
\begin{lstlisting}
void scan_matrix(double **a, int size);
\end{lstlisting}

\item Вывод на стандратный поток матрицы-столбца
\begin{lstlisting}
void print_vect(double *a, int size);
\end{lstlisting}

\item Ввод матрицы-столбца со стандратного потока 
\begin{lstlisting}
void scan_vect(double *a, int size);
\end{lstlisting}

\end{itemize}

\end{itemize}

\newpage
\section{Тестирование}

\begin{itemize}
\item Часть 1\\
Тестирование проводится на наборах СЛАУ из приложения 1 и примера 1 приложения 2. \\
Для матриц, имеющих определитель 0, программа выводит сообщение об этом на стандартный поток ошибок и завершается с кодом 1, кроме того метод верхней реалксации применим только для положительно определенных матриц.\\
Результат работы на невырожденных матрицах сравниваем с точным ответом, полученном на сайте wolframalpha.com.\\
Для генерации теста из приложения 2 использована программа:
\begin{lstlisting}
int main(int argc, char**argv)
{
    if (!strcmp(argv[1], "--frm")){
        double n = 20, m = 8;
        printf("%d\n", (int)n);
        for (int i = 1; i <= n; i++){
            for (int j = 1; j <= n; j++){
                if (i == j){
                    printf("%f ", n + m*m + j/m + i/n);
                } else {
                    printf("%f ", (i + j) / (m + n));
                }
            }
            printf("\n");
        }
        for (int i = 1; i <= n; i++){
           printf("%f ", 200 + 50 * i);
        }

    }
    printf("\n");
    return 0;
}
\end{lstlisting}


Для тестирования запускаем программу с ключом --gauss или --gauss+, в качетсве результата получаем определитель матрицы из левой части, обратную матрицу, число обусловленности и вектор - решение СЛАУ.

Например, для варианта 3 - 1 создадим файл 1.txt:
\begin{tabular}{ccccccc}

2 & 5 & 4 & 1\\
1 & 3 & 2 & 1\\
2 & 10 & 9 & 7\\
3 & 8  & 9 & 2\\
20 & 11  & 40 & 37\\

\end{tabular}

Запустив программу с ключом - -gauss+ с перенаправлением стандартного ввода-вывода, получим в качестве результата:\\
Определитель: -3.000\\
Обратная матрица:
\begin{tabular}{ccccccc}

15.000 & -21.000 & 2.000 & -4.000\\
-6.667 & 10.333 & -1.000 & 1.667\\
-0.333 & -0.333 & 0 & 0.333\\
5.667 & -8.333  & 1 & 1.667\\
\end{tabular}

Число обусловленности: 210.000\\
Решение: 
\begin{tabular}{ccccccc}
15.000 \\
-6.667 \\
-0.333 \\
5.667 \\
\end{tabular}

Для варианта 3 - 2 программа выдает сообщение о том, что матрица вырождена.


\item Часть 2\\

Для варианта 3 - 1 получаем те же ответы, что и при использовании метода Гаусса.\\
В варианте 3 - 2 программа выдает сообщение о том, что матрица вырождена.\\
В варианте 3 - 3 матрица не является положительно определенной\\

Исследуем скорость сходимости метода верхней релаксации.

Заметим, что для сходимости метода верхней релаксации необходима симметричность матрицы-левой части. Поэтому для тестирования будем использовать формулы из приложения 2, пример 1 c параметром max-iter = 500, epsilon = 0.000001. Запускаем тесты при различных значениях $test$.
Оценим скорость сходимости при различных w:
\begin{center}
\begin{tabular}{ccccccc}
w & тест 1 & тест 2 & тест 3 & тест 4 & тест 5 & тест 6\\
0.2 & 47 & 77 & 76 & 96 & 70 & 75\\
0.4 & 23 & 36 & 34 & 44 & 32 & 35\\
0.6 & 14 & 21 & 19 & 25 & 19 & 20\\
0.8 & 8  & 12 & 12  & 15 & 11 & 12\\
1.0 & 5  & 7   & 7   &  7  & 5 & 6\\
1.2 & 10 & 14 & 14 & 16 & 12 & 13\\
1.4 & 17 & 24 & 24 & 28 & 21 & 23\\
1.6 & 29 & 43 & 43 & 50 & 37 & 40\\
1.8 & 66 & 97 & 97 & 113 & 84 & 91\\
\end{tabular}
\end{center}
По полученным измерениям можно  сказать, что максимальная скорость сходимости достигается при $w = 1$
Дополнительно можно удедится, что для не симметричных матриц метод быстро расходится и все координаты обращаются в бесконечность.

\end{itemize}


\section{Выводы}

\begin{itemize}
\item Часть 1\\
Можно заметить, что на всех тестах оба метода Гаусса или сходятся одновременно, и причём с почти одинаковой точностью, или расходятся одновременно, из чего можно сделать вывод, что значительной разницы между методами нет.

\item Часть 2\\
Можно сделать однозначный вывод, что в случае симметрической матрицы метод верхней релаксации будет сходится крайне быстро, особенно если выбрать $w = 1$. В этом случае метод верхней релаксации будет давать решение с высокой точностью уже всего через $10-20$ итераций, что будет давать значительную выгоду по сравнению с методами Гаусса. Однако множество применимости этого метода сильно уже, чем у методов Гаусса.


\end{itemize}

\newpage


\end{document}